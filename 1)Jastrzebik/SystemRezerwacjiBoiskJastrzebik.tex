\documentclass[a4paper, portrait,11pt]{article}
\usepackage[verbose,a4paper,tmargin=3cm,bmargin=3cm,lmargin=2.5cm,rmargin=2.5cm]{geometry}
\usepackage[utf8]{inputenc}
\usepackage{polski}
\usepackage{amsmath}
\usepackage{amsfonts}
\usepackage{amssymb}
\usepackage{lastpage}
\usepackage{indentfirst}
\usepackage{verbatim}
\usepackage{graphicx}
\usepackage{fancyhdr}
\usepackage{listings}
\frenchspacing
\pagestyle{fancyplain}
\fancyhf{}
\renewcommand{\headrulewidth}{0pt}
\renewcommand{\footrulewidth}{0.4pt}
\newcommand{\degree}{\ensuremath{^{\circ}}} 
\fancyfoot[L]{PSI: Justyna Hubert i Karol Podlewski}
\fancyfoot[R]{\thepage\ / \pageref{LastPage}}

\begin{document}

\begin{titlepage}
\begin{center}
\begin{tabular}{rl}
\begin{tabular}{|r|}
\hline \\
\large{\underline{210200~~~~~~~~~~~~~~~~~~~~~~~~~~~~~~~~~~~~~~~~~~~} }\\
$^{numer\ indeksu}$\\
\large {\underline{Justyna Hubert~~~~~~~~~~~~~~~~~~~~~~~~~~~~~~~} }\\
$^{imie\ i\ nazwisko}$ \\\\ \hline
\end{tabular} 
&
\begin{tabular}{|r|}
\hline \\
\large{\underline{210294~~~~~~~~~~~~~~~~~~~~~~~~~~~~~~~~~~~~~~~~~~~} }\\
$^{numer\ indeksu}$\\
\large {\underline{Karol Podlewski~~~~~~~~~~~~~~~~~~~~~~~~~~~~~~~} }\\
$^{imie\ i\ nazwisko}$ \\\\ \hline
\end{tabular} 

\end{tabular}
~\\~\\~\\ 
\end{center}
\begin{tabular}{ll}
\LARGE{\textbf{Data}}& \LARGE{2018-10-12}\\
\LARGE{\textbf{Kierunek}}& \LARGE{Informatyka}\\
\LARGE{\textbf{Rok akademicki}}& \LARGE{2018/19} \\
\LARGE{\textbf{Semestr}}& \LARGE{5} \\
\LARGE{\textbf{Specjalizacja}}& \LARGE{IOAD} \\
\LARGE{\textbf{Grupa dziekańska}}& \LARGE{3} \\\\\\\\\\
\end{tabular}

\begin{center}
\textbf{\LARGE{\\~\\Projektowanie Systemów\\Informatycznych\\~\\~\\}}
\end{center}


\begin{center}
\textbf{\Huge{System rezerwacji boisk sportowych:\\~\\ JASTRZĘBIK}} \\~\\
\end{center}

\end{titlepage}
\setcounter{page}{2}

\section {Wprowadzenie}

\subsection {Cel dokumentu}
Celem dokumentu jest przeprowadzenie analizy i określenie specyfikacji systemu umożliwiającego rezerwowanie boisk sportowych. Zaprezentowane zostały opisy wymaganych funkcjonalności oraz przypisane uprawnienia dla danych użytkowników systemu.

\subsection {Zakres produktu}
System skierowany będzie do dwóch grup użytkowników:
\begin{itemize}
	\item Wszystkich osób powyżej 12 roku życia, zamieszkałych lub przebywających na terenie Polski, zainteresowanych wynajęciem boiska sportowego w Łodzi w celu aktywnego spędzenia czasu. Otrzymają ograniczony dostęp do funkcjonalności systemu.
	\item Animatorów sportowych sprawujących pieczę nad danym boiskiem. Będą oni zarządzać system i otrzymają pełen zakres uprawnień.
\end{itemize}

\subsection {Definicje, akronimy i skróty}
Stroną udostępniającą wynajęcie boiska sportowego w Łodzi będzie Państwo Polskie.
Stroną rezerwującą boisko sportowe będzie osoba powyżej 12 roku życia. 

\subsection {Odwołania do literatury}
Przykładem istniejącego portalu umożliwiającego rezerwację boiska jest ostroleka.pl/boiska

\subsection {Omówienie dokumentu}
Dokument opisuje w szczegółowy sposób zadania i wymagane funkcjonalności systemu. Składa się z czterech części.
\begin{itemize}
	\item Część pierwsza  została podzielona na pięć podsekcji i poświęcona jest przedstawieniu idei systemu. Ma ona za zadanie przygotować do zaznajomienia się z podstawowymi definicjami związanymi z systemem. Wyjaśnia jaki jest cel dokumentu, do kogo skierowany jest produkt, objaśnia dane definicje oraz akronimy, na przykładzie źródła elektronicznego odwołuje się do istniejącego już portalu umożliwiającego rezerwację boisk. 
	\item Druga część dokumentu zawiera ogólny zarys systemu Jastrzębik. Część ta została podzielona na pięć podsekcji, które objaśniają kontekst funkcjonowania produktu, a także jego główne funkcje, przedstawiają charakterystykę użytkowników oraz nałożone ograniczenia wraz z założeniami i zależnościami występującymi w systemie.
	\item Część trzecia poświęcona jest wymaganiom funkcjonalnym występującym w systemie. Opisuje czynności i operacje wykonywane przez priodukt.
	\item Część czwarta przedstawia wymagania niefunkcjonalne pojawiające się w systemie. Opisuje ograniczenia, przy których produkt powinien realizować swoje funkcje.
\end{itemize}

\section {Opis ogólny}

\subsection {Kontekst funkcjonowania}
System Jastrzębik umożliwiać będzie rezerwację wybranego boiska sportowego znajdującego się na terenie miasta Łodź. W celu zarejestrowania się w systemie użytkownik będzie zobowiązany do podania swojego imienia, nazwiska, numeru PESEL, miejsca zamieszkania oraz numeru telefonu lub adresu mailowego. Po wykonaniu uwierzytelnienia użytkownik będzie miał możliwość złożenia elektronicznego wniosku wynajęcia danego boiska sportowego o wybranej porze. Wniosek rozpatrywany będzie zgodnie z regulaminem obiektu, wg daty złożenia wniosku i spełnieniem wymogów formalnych, bezpośrednio przez animatorów przypisanych do danych boisk sportowych. Pierwszeństwo rezerwacji otrzymają osoby korzystające z boisk w ubiegłych miesiącach. Animatorzy będą upoważnieni do wglądu w szereg szczegółowych informacji o rezerwacjach, rezerwujących oraz będą posiadali możliwość wprowadzania zmian w harmonogramie boisk. 

\subsection {Charakterystyka użytkowników}
Docelowymi użytkownikami rezerwującymi boiska sportowe są osoby powyżej 12 roku życia zamieszkałe lub przebywające na terenie Polski. Będą oni posiadali ograniczone uprawnienia w systemie - możliwość złożenia wniosku o wynajęcie danego boiska i ewentualne późniejsze jego edytowanie lub anulowanie.
\\\indent
Użytkownikami odpowiedzialnymi za rozpatrywanie wpływających rezerwacji będą wcześniej wybrani animatorzy boisk. Będą oni posiadali pełne uprawnienia w systemie - zatwierdzanie lub odrzucanie rezerwacji, uaktualnianie informacji na temat boisk, ustalanie harmonogramu dostępności boisk.

\subsection {Główne funkcje produktu tu jakos musze ogarnac te kropki chyba}
\begin{itemize}
	\item Zarejestrowanie się w systemie.
	\item Możliwość złożenia elektronicznego wniosku o rezerwację boiska sportowego.
	\item Możliwość złożenia elektronicznego wniosku o rezerwację boiska sportowego.
	\item Sprawdzanie stanu złożonej wcześniej rezerwacji.
	\item Modyfikacja rezerwacji - edytowanie, anulowanie wniosku.
	\item Wgląd do harmonogramu dostępności boisk.
\end{itemize} 

\subsection {Ograniczenia}
System musi spełniać założenia zgodne z rozporządzeniem Parlamentu Europejskiego i Rady (UE) 2016/679 z 27.04.2016 r. w sprawie ochrony osób fizycznych w związku z przetwarzaniem danych osobowych i w sprawie swobodnego przepływu takich danych oraz uchylenia dyrektywy 95/46/WE (ogólne rozporządzenie o ochronie danych) (Dz. U. UE. L. z 2016 r. Nr 119, str. 1).

\subsection {Założenia i zależności}
\begin{itemize}
	\item Warunkiem złożenia elektronicznej rezerwacji będzie 			wcześniejsze zarejestrowanie się w systemie, a następnie uwierzytelnienie. Tylko osoby, który ukończyły 12 rok życia będą mogły się zarejestrować
	\item Nie będzie możliwe zarezerwowanie boiska, jeśli w danym momencie jest już wynajęte.
	\item Nie będzie możliwe zarezerwowanie boiska w godzinach 22-6.
	\item Nie będzie możliwe zarezerwowanie boiska w sezonie zimowym 1.12 - 28.02.
	\item Dwukrotna nieobecność podczas zarezerwowanego terminu będzie skutkować zablokowaniem możliwości rezerwacji boisk na okres 3 miesięcy. 
\end{itemize}

\section {Wymagania funkcjonalne}
\begin{enumerate}
	\item Rejestracja użytkowników.
	\item Logowanie oraz wylogowanie się z systemu.
	\item Wgląd do harmonogramu dostępności boisk.
	\item Modyfikacja rejestracji.
	\item Anulowanie rejestracji.
	\item Rezerwacja wybranego boiska.
	\item Dostęp do zdjęć dostępnych boisk sportowych.
	\item Dostęp do panelu kontaktu - możliwość elektronicznego skontaktowania się z animatorami.
	\item 
\end{enumerate}

\section {Wymagania niefunkcjonalne}
\begin{enumerate}
	\item Użytkownik nie ma możliwości wglądu kont innych użytkowników.
	\item Uzytkownik podczas rejestracji musi mieć ukończone co najmniej 12 lat.
	\item Użytkownik nieuprzywilejowany nie może wprowadzać zmian w rejestracjach innego użytkownika.
	\item 
\end{enumerate}

\end{document}